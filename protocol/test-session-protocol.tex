\documentclass[]{article}
\usepackage{lmodern}
\usepackage{amssymb,amsmath}
\usepackage{ifxetex,ifluatex}
\usepackage{fixltx2e} % provides \textsubscript
\ifnum 0\ifxetex 1\fi\ifluatex 1\fi=0 % if pdftex
  \usepackage[T1]{fontenc}
  \usepackage[utf8]{inputenc}
\else % if luatex or xelatex
  \ifxetex
    \usepackage{mathspec}
  \else
    \usepackage{fontspec}
  \fi
  \defaultfontfeatures{Ligatures=TeX,Scale=MatchLowercase}
\fi
% use upquote if available, for straight quotes in verbatim environments
\IfFileExists{upquote.sty}{\usepackage{upquote}}{}
% use microtype if available
\IfFileExists{microtype.sty}{%
\usepackage{microtype}
\UseMicrotypeSet[protrusion]{basicmath} % disable protrusion for tt fonts
}{}
\usepackage[margin=1in]{geometry}
\usepackage{hyperref}
\hypersetup{unicode=true,
            pdftitle={Testing session protocol},
            pdfauthor={Yiming Qian, Andrea Seisler, \& Rick Gilmore},
            pdfborder={0 0 0},
            breaklinks=true}
\urlstyle{same}  % don't use monospace font for urls
\usepackage{graphicx,grffile}
\makeatletter
\def\maxwidth{\ifdim\Gin@nat@width>\linewidth\linewidth\else\Gin@nat@width\fi}
\def\maxheight{\ifdim\Gin@nat@height>\textheight\textheight\else\Gin@nat@height\fi}
\makeatother
% Scale images if necessary, so that they will not overflow the page
% margins by default, and it is still possible to overwrite the defaults
% using explicit options in \includegraphics[width, height, ...]{}
\setkeys{Gin}{width=\maxwidth,height=\maxheight,keepaspectratio}
\IfFileExists{parskip.sty}{%
\usepackage{parskip}
}{% else
\setlength{\parindent}{0pt}
\setlength{\parskip}{6pt plus 2pt minus 1pt}
}
\setlength{\emergencystretch}{3em}  % prevent overfull lines
\providecommand{\tightlist}{%
  \setlength{\itemsep}{0pt}\setlength{\parskip}{0pt}}
\setcounter{secnumdepth}{0}
% Redefines (sub)paragraphs to behave more like sections
\ifx\paragraph\undefined\else
\let\oldparagraph\paragraph
\renewcommand{\paragraph}[1]{\oldparagraph{#1}\mbox{}}
\fi
\ifx\subparagraph\undefined\else
\let\oldsubparagraph\subparagraph
\renewcommand{\subparagraph}[1]{\oldsubparagraph{#1}\mbox{}}
\fi

%%% Use protect on footnotes to avoid problems with footnotes in titles
\let\rmarkdownfootnote\footnote%
\def\footnote{\protect\rmarkdownfootnote}

%%% Change title format to be more compact
\usepackage{titling}

% Create subtitle command for use in maketitle
\providecommand{\subtitle}[1]{
  \posttitle{
    \begin{center}\large#1\end{center}
    }
}

\setlength{\droptitle}{-2em}

  \title{Testing session protocol}
    \pretitle{\vspace{\droptitle}\centering\huge}
  \posttitle{\par}
    \author{Yiming Qian, Andrea Seisler, \& Rick Gilmore}
    \preauthor{\centering\large\emph}
  \postauthor{\par}
      \predate{\centering\large\emph}
  \postdate{\par}
    \date{2019-11-12 11:39:14}


\begin{document}
\maketitle

{
\setcounter{tocdepth}{3}
\tableofcontents
}
\subsection{Before participant
arrives}\label{before-participant-arrives}

\begin{itemize}
\tightlist
\item
  Check to see if there have been any cancellations.
\item
  If the scheduled study is still on the books, proceed as follows.
\end{itemize}

\subsubsection{Set-up for Vision
Screening}\label{set-up-for-vision-screening}

\paragraph{Preparation}\label{preparation}

Materials for vision screening are stored on the table next to Andrea's
office.

\begin{itemize}
\tightlist
\item
  Make sure the black tape is on the floor 10ft from the HOVT Snellen
  Acuity Chart which is on the door to 503B
\item
  Place Stereo Acuity Test and Glasses on table
\item
  Place Color Vision Test on table
\item
  Place the Vision Screening Score Sheet on the table
\end{itemize}

\paragraph{Review vision screening
procedures}\label{review-vision-screening-procedures}

The vision screening protocol may be reviewed at
\href{vision-screening-protocol.html}{this link}

\subsubsection{Set up for computer-based
tasks}\label{set-up-for-computer-based-tasks}

\paragraph{Stimuli Computer}\label{stimuli-computer}

\begin{itemize}
\item
  \emph{Log into Data Collection Computer}
\item
  Turn on the power of the data collection computer
\item
  Turn on the CRT monitor in 503B
\item
  Log-in (Gilmore Lab)
\item
  \emph{Start Psychopy}
\item
  Click \textbf{PsychoPy} icon on Task Bar
  \includegraphics{images/PsychoPy-1.PNG}\\
\item
  \emph{Double-check monitor settings within Windows}
\item
  Click Settings (`gear') icon on Task Bar
  \includegraphics{images/DispSettings-1.PNG}\\
\item
  Choose \textbf{System}\\
  \includegraphics{images/DS2.PNG}\\
\item
  Choose \textbf{Display}\\
  \includegraphics{images/ds3.PNG}\\
\item
  Choose \textbf{Advanced display settings} (You may need to scroll down
  to see this)\\
  \includegraphics{images/DS4.PNG}\\
\item
  Make sure the window that appears has the following Settings\\
  \includegraphics{images/ds5.PNG}\\
\item
  \emph{Double-check Brightness/Contrast of monitor}
\item
  Contrast:
\item
  Brightness:
\item
  Press any button on the monitor (except Signal A/B/OSD OFF and the
  Power button)
\item
  Navigate to the leftmost option in the settings menu (looks like a
  half moon)
\item
  Press the down button on the monitor
\item
  Adjust the Contrast (leftmost option) to the required setting using
  the +/- buttons on the monitor
\item
  Adjust the Brightness (second option from the left) to the required
  setting using the +/- buttons on the monitor
\item
  \emph{Check monitor within PsychoPy}\\
\item
  Go to \textbf{Monitor Settings}\\
  \includegraphics{images/pp2.png}\\
\item
  View Settings, they should be as follows\\
  \includegraphics{images/pp3.PNG}
\end{itemize}

\paragraph{Survey Computer}\label{survey-computer}

\begin{itemize}
\tightlist
\item
  Log-in to survey computer
\item
  Load page with surveys:
  \url{https://pennstate.qualtrics.com/jfe/form/SV_0Cad5AtrbQN0GKV}
\end{itemize}

\subsubsection{After participant
arrives}\label{after-participant-arrives}

\paragraph{Welcome participant}\label{welcome-participant}

Say:

\begin{quote}
``\emph{Welcome to the brain, behavior, and development lab. Are you
here for the study about individual difference of motion perception?}''
\end{quote}

If the participant answers yes, say:

\begin{quote}
``\emph{Great. You can put your coat on the back of the door and your
bag here.}''
\end{quote}

\begin{itemize}
\tightlist
\item
  Store coat on back of main door and bags by the file/bookcase.
\end{itemize}

\begin{quote}
``\emph{Are you \textless{}NAME OF PERSON ON SONA SYSTEMS SITE SCHEDULED
FOR THIS SESSION\textgreater{}?}''
\end{quote}

\begin{itemize}
\tightlist
\item
  If the participant answers yes, say:
\end{itemize}

\begin{quote}
``\emph{Ok. We want to make sure that you get credit for participation.
Please sit here for the first portion of the study.}''
\end{quote}

\begin{itemize}
\tightlist
\item
  Have the person sit at the computer where the survey will be taken.
\end{itemize}

\paragraph{Begin the survey}\label{begin-the-survey}

\begin{itemize}
\tightlist
\item
  You will see the Participant ID on the top of implied consent. Take a
  note of this participant ID (only the numeric code without comma or
  hyphen).
\end{itemize}

Conduct the implied verbal consent. You may say to the participant or
have them read the following text:

You are being invited to participate in a research study.

\begin{itemize}
\tightlist
\item
  The purpose of this voluntary research study is to investigate how
  human beings perceive motion in an experimental setting. The results
  of this research study will help scientists gain a deeper
  understanding of what contributes to individual differences in motion
  perception, and whether or how motion perception is correlated with
  other aspects of life.
\item
  You will complete one computer-based surveys. Then, you will complete
  two short computer tasks in which you will attempt to detect motion on
  a computer screen.
\item
  All questionnaire and computer task data you provide will be saved
  using a numeric code. No information about your identity or how to
  contact you will be saved with the data.
\item
  If you are participating as part of the Psychology Subject Pool, you
  will receive course credit for participation as specified in the
  syllabus provided by your instructor. This means you will get 1 credit
  for participating this research. Alternative means for earning this
  course credit are available as specified in the syllabus.
\end{itemize}

If you have questions, complaints, or concerns about the research, you
could contact principal investigator Yiming Qian or her advisor Rick
Gilmore. If you have questions regarding your rights as a research
subject or concerns regarding your privacy, you could contact the Office
for Research Protections.

Your participation is voluntary and you may decide to stop at any time.
You do not have to answer any questions that you do not want to answer.

Clicking the ``Next'' button implies two things: (1) that you are at
least 18 years of age, and (2) you voluntarily consent to participate in
the research. Thank you!

Once the consent is complete (It means the participant clicks to the
next page), say:

\begin{quote}
``\emph{That's great. Now we'd like to move on to the vision screening
portion of the study. Could you stand behind this line?}''
\end{quote}

\subsubsection{Complete pattern visual acuity
testing}\label{complete-pattern-visual-acuity-testing}

\begin{itemize}
\tightlist
\item
  Complete \href{vision-screening-protocol.html}{pattern acuity test}
\item
  Adult - HOTV @ 10ft
\end{itemize}

\subsubsection{Questionnaires}\label{questionnaires}

\begin{quote}
``\emph{Thank you. Now we'd like to move on to the questionnaire portion
of the study. Please sit down. You can follow the instructions and
finish the survey. Feel free to ask me if you have any questions. And
let me know when you finish it.}''
\end{quote}

\begin{itemize}
\item
  Have the participant sit back down at the computer.
\item
  Let the participants finish the questionnaire.
\item
  Answer the questions if the participants have any, when they works on
  the questionnaires. But in careful in the hobby page, spatial and
  verbal page, because the time are recorded. The page will vanish when
  the time have passed. So, depending on the nature of the questions,
  answer them fast and emphasize the time is recorded in this page.
\item
  close the door for participants After answering the question, say:
  \textgreater{}``\emph{Beware: there is a time limit for this page. }''
\item
  If the participants have questions in the instruction page of hobby
  test, spatial and verbal test, answer careful and make sure the
  participants understand well.
\item
  After the participants finish the questionnaire, ask them if they need
  a little break. If the participant wants to keep going, lead them to
  the test room
\end{itemize}

Say:

\begin{quote}
``\emph{You have finished the first part of testing. Next you have
behavorial testing. Do you want to continue or have a little break?}''
\end{quote}

\subsubsection{Set-up for computer-based
tasks}\label{set-up-for-computer-based-tasks-1}

\begin{itemize}
\tightlist
\item
  Guide participant to the testing room.
\item
  Have them sit in the chair.
\item
  Adjust the monitor and participant position.
\item
  The monitor should be located \textbf{60cm} from the bridge of the
  nose on the participant.
\item
  The chair height should be set so the participant is looking in the
  middle of the screen.
\item
  Guide the participant to use the arrow keys for responses and the
  space bar to advance the screen.
\end{itemize}

\subsubsection{Run computer-based tasks}\label{run-computer-based-tasks}

\paragraph{Select run order}\label{select-run-order}

The order of the computer experiments will be randomized across
participants based on the participant ID shown in Qualtrics - run the
temporal duration threshold task first (Murray et al.) if the ID number
is even. - run the contrast sensitivity task (Abramov et al.) first if
the ID number is odd. \emph{Record the task run first on the experiment
run log.}

\paragraph{Temporal duration threshold task (Murray et
al.)}\label{temporal-duration-threshold-task-murray-et-al.}

\begin{itemize}
\tightlist
\item
  Open PsychoPy by clicking on the icon located on the desktop.
  \includegraphics{images/PsychoPy-1.PNG}\\
\item
  When PsychoPy opens, open the file for this experiment.

  \begin{itemize}
  \tightlist
  \item
    From the \texttt{File} menu, select the \texttt{Open\ Recent...}
    command and select the \texttt{motion-temporal-threshold.py} file.
  \end{itemize}
\item
  When the file opens, run the experiment by pressing press the green
  (running person) button. \includegraphics{images/PPrunningMan.png}

  \begin{itemize}
  \tightlist
  \item
    \textbf{Be careful not to type in the programming window.}
  \end{itemize}
\item
  Experimenters need to fill in the participant ID and gender.

  \begin{itemize}
  \tightlist
  \item
    A pop-up window will appear.
  \item
    Participant ID in the pop-up window have shown ``YYYYMMDD'', which
    is the first part of participant ID. Enter the rest numbers of
    participant ID based on the note you take from the beginning of
    qualtrics.
  \item
    Enter gender (enter ``f'' or ``m'', no upper case) in the pop
    window, and press the \texttt{Ok} button to enter the data.
  \end{itemize}
\item
  Speak to the participant
\end{itemize}

\begin{quote}
``In this task, you need to detect the moving direction of a small patch
of stripes. The time the patch appears on the display will get shorter
and shorter. Our goal is to find out the shortest duration you need to
detect the direction of motion.''
\end{quote}

\begin{quote}
``Which hand do you prefer to press the arrow keys?'' ``Put your fingers
on the left and right arrow keys. You'll press the left arrow if you see
motion to the left and the right arrow if you see motion to the right.
If you aren't sure, make your best guess.'' For the left-handed: ``You
could press this ENTER key on the right side to preceed instead of space
bar.''
\end{quote}

\begin{quote}
``On the computer screen, you will see a black dot at first. When the
black dot appear, press the space bar to start the trial. Then you will
see the patch. Make responses of left or right when the white dots
appear.
\end{quote}

\begin{quote}
``Remember, accuracy is more important that speed. Please take your
time.''
\end{quote}

\begin{quote}
``This task takes about 1 min to complete. But to make sure that we get
reliable results, there are 4 sections. You can take a short break
between the sections.''
\end{quote}

\begin{quote}
``Do you have any questions right now? Okay. I will leave you in the
room. Follow the instructions on the screen. Call me when you finished
this part.''
\end{quote}

\paragraph{Contrast sensitivity task (Abramov et
al.)}\label{contrast-sensitivity-task-abramov-et-al.}

\includegraphics{images/PsychoPy-1.PNG}\\
- When PsychoPy opens, open the file for this experiment. - From the
\texttt{File} menu, select the \texttt{Open\ Recent...} command and
select the \texttt{xxx.py} file. - When the file opens, run the
experiment by pressing press the green (running person) button.
\includegraphics{images/PPrunningMan.png}\\
- \textbf{Be careful not to type in the programming window.} -
Experimenters need to fill in the participant ID and gender. - A pop-up
window will appear. - Participant ID in the pop-up window have shown
``YYYYMMDD'', which is the first part of participant ID. Enter the rest
numbers of participant ID based on the note you take from the beginning
of qualtrics. - Enter gender (enter ``f'' or ``m'', no upper case) in
the pop window, and press the \texttt{Ok} button to enter the data. -
Speak to the participant

\begin{quote}
``You will see a small patch of black and white stripes which is
horizontal or vertical.Be careful. You need to detect the direction of
the stripes not the moving direction. You can press the LEFT or RIGHT
buttons if you see the stripes are horizontal, UP or DOWN button if you
see the stripes are vertical. But if you aren't sure, just guess.''
\end{quote}

\begin{quote}
``In this task, you will see a small patch of black and white stripes
and need to detect the direction of these strips, rather than their
moving direction. The luminance of the stripes will get smaller and
smaller. Our goal is to find out the smallest luminance that you need to
detect the direction of stripes.''
\end{quote}

\begin{quote}
``Which hand do you prefer to press the arrow keys?'' ``Put your fingers
on the left and down arrow keys. You'll press the left arrow if you see
see horizontal stripes and the down arrow if you see vertical stripes.
If you aren't sure, make your best guess.'' For the left-handed: ``You
could press this ENTER key on the right side to preceed instead of space
bar.''
\end{quote}

\begin{quote}
``Remember, accuracy is more important that speed. Please take your
time.''
\end{quote}

\begin{quote}
``This task takes about 1 min to complete. But to make sure that we get
reliable results, there are 4 sections. You could take a short break
between the sections.''
\end{quote}

\begin{quote}
``Do you have any questions right now? Okay. I will leave you in the
room. Follow the instructions on the screen. Call me when you finish
this part.''
\end{quote}

\subsubsection{(Optional) stereo acuity and color vision
tests}\label{optional-stereo-acuity-and-color-vision-tests}

If there is time left (5 min before the end of the 1 hr session),

\begin{quote}
``\emph{Thank you so much. It looks like we have time for two more short
vision tests. Please come sit over here at this table.}''
\end{quote}

Escort participant to table.

\begin{itemize}
\tightlist
\item
  Complete \href{vision-screening-protocol.html}{Stereo Acuity Test}
\item
  Complete \href{vision-screening-protocol.html}{Color Vision Test}
\end{itemize}

\subsection{After session ends}\label{after-session-ends}

\subsubsection{Thank participant}\label{thank-participant}

\begin{itemize}
\tightlist
\item
  After the participant finishes all the tests, thank him/her.
\end{itemize}

\begin{quote}
``\emph{Thank you for participating this experiment. We appreciate your
time. Do you have any questions?}''
\end{quote}

\begin{itemize}
\tightlist
\item
  Answer any questions the participant might have. You may direct them
  to Yiming or to Dr.~Gilmore if you are unable to answer the question.
\end{itemize}

\subsubsection{Give participant credit on
SONA}\label{give-participant-credit-on-sona}

\begin{itemize}
\tightlist
\item
  Yiming or Andrea will assign credit in SONA.
\end{itemize}

\subsubsection{Clean-up}\label{clean-up}

\begin{itemize}
\tightlist
\item
  Clean keyboard, mouse and table and begin
  \href{sex-differences-data-export.md}{data export} (separate
  protocols).
\end{itemize}


\end{document}
